%
% Project plan for degree project
% Template version: Don't forget to update when updating the template
% Update also the Word template. Keep the version numbers in both formats in sync.
\newcommand{\theVersion}{1.6 -- June 12, 2020}
%
\documentclass[12pt,a4paper,twoside]{article}
\usepackage{times}
%\usepackage{mathptmx}
\usepackage{multirow}
\usepackage{hyperref}
\usepackage[utf8]{inputenc}
\usepackage[T1]{fontenc}
\usepackage[top=2.5cm,bottom=2.5cm,left=2.5cm,right=2.5cm]
{geometry}
% --------------------------------------------
% package "todonotes" is used for notes and comments
% To disable all notes you can use the option "disable", see second row below:
\usepackage[color=blue!10,textsize=footnotesize,textwidth=25mm]{todonotes}
%\usepackage[disable]{todonotes} %passive=do not show
% --------------------------------------------
\usepackage[sort&compress]{natbib}
\setcitestyle{numbers,square,comma}
\usepackage{enumitem}
\setlist{topsep=1ex,itemsep=0.5ex,parsep=0pt,partopsep=0pt}  % this sets the vertical spacing between list items and surrounding parapgraphs

% Redefine the course code and name.
\newcommand{\theCourse}{DV2572: Masters thesis course in Computer Science}

% Choose one of the following for the purpose of the submission.
%\newcommand{\thePurpose}{Project idea} % Use for a project idea
\newcommand{\thePurpose}{Project plan} % Use for a project plan

% Redefine the company name as a variable for easier reference.
\newcommand{\theCompany}{CompanyOrganizationName}

% *** Do not touch the following lines. BEGIN. ***
\title{\thePurpose\ for degree project\\\vspace{1mm}\small{Template version \theVersion}}
\author{\textsc{\theCourse}}
\date{\today}
\begin{document}
\maketitle
\vspace*{-5mm}
% *** END. Do not touch the lines above. ***

% *** PLEASE START HERE ***
\noindent % Do not delete this line or add an empty line below.
\begin{tabular}{|l|l|p{9.3cm}|}
\hline
Title      		  & \multicolumn{2}{|p{12cm}|}{Behaviour-based detection of ransomware attacks in the Cloud using machine learning} \\
\hline
Classification      		  & \multicolumn{2}{|l|}{Security and Privacy--Artificial Intelligence--Cloud Computing} \\
\hline\hline
\multirow{4}{*}{Student 1}
 & Name           & Yaroslav Popryho\\\cline{2-3}
 & E-Mail         & yapo22@student.bth.se \\\cline{2-3}
 & Person nr      & 991102T476 \\\cline{2-3}
 & Program        & Computer Science \\\cline{2-3}
\hline
\multirow{4}{*}{Student 2}
 & Name           & Leonid Popryho\\\cline{2-3}
 & E-Mail         & lepo22@student.bth.se \\\cline{2-3}
 & Person nr      & 991102T450 \\\cline{2-3}
 & Program        & Computer Science \\\cline{2-3}
\hline
\multirow{3}{*}{Supervisor}
 & Name \& title  & Oleksandr Adamov, Senior lecturer \\\cline{2-3}
 & E-Mail         &  oleksandr.adamov@bth.se \\\cline{2-3}
 & Department     &  Department of Software Engineering \\
\hline

\end{tabular}


\section{Introduction}
\label{sec:intro}
Ransomware attacks pose a significant threat to digital information, and with the increasing adoption of cloud storage services, attackers now target cloud environments (Kolodenker et al., 2017). \cite{kolodenker2017ezdec}

The existing literature on ransomware detection has primarily focused on local environments, employing a wide range of machine learning techniques, such as supervised learning, unsupervised learning, and reinforcement learning, to classify and identify potential threats (Sgandurra et al., 2016). \cite{sgandurra2016automated} A recent work by our supervisor, titled "Reinforcement Learning for Anti-Ransomware Testing," \cite{Adamov} demonstrates the potential of reinforcement learning to detect ransomware attacks in local systems. However, there is a limited body of research on applying these approaches to the cloud environment, which presents unique challenges and requirements.

In this thesis, we aim to develop a behavior-based ransomware detection system for cloud environments, specifically focusing on Google Drive, using machine learning techniques. We will create a dedicated Google Workspace and utilize the Google Cloud Platform for developing the anomaly detection classifier.

We will review related work in ransomware detection and machine learning approaches (e.g., Zhu et al., 2020; Cabaj et al., 2018), \cite{zhu2019ransomware} \cite{cabaj2018machine} to select suitable techniques for our research. Our anomaly detection classifier will analyze user activities in the cloud, such as file access patterns and permission changes, to detect deviations indicative of ransomware attacks (Al-Rimy et al., 2019). \cite{alrimy2019ransomclassifier}

We will validate our system's performance by conducting experiments in our Google Workspace, emulating ransomware attacks, and comparing the classifier's performance against existing techniques. Our thesis aims to contribute a novel, behavior-based detection system for ransomware attacks in cloud environments, advancing the state-of-the-art and providing a scalable solution for various cloud storage providers.


\subsection{Ethical, societal and sustainability aspects}
\label{sec:ethics}


\subsection{Ethics}\label{sec:Ethics}

The proposed research will be conducted in accordance with ethical principles and guidelines for academic research. The data used in the study will be properly anonymized to protect the privacy of individuals, and all necessary permissions and approvals will be obtained before conducting the research. The study will also be conducted with the aim of avoiding any harm or negative impact on the society or individuals.


\subsection{Society}\label{sec:Society}

The proposed research is expected to have a positive impact on society, by contributing to the development of suitable approach for detecting ransomware attacks in the cloud environment. Ransomware attacks cause significant disruptions to normal operations and often result in financial losses, and the proposed research has the potential to mitigate these impacts by detecting and preventing such attacks.

\subsection{Sustainability}\label{sec:Sustainability}

The proposed research does not have direct implications for sustainability, as it does not deal with environmental issues. However, the development of more effective methods for detecting and preventing ransomware attacks in the cloud environment may contribute to the overall sustainability of cloud systems, by reducing downtime and ensuring the smooth functioning of critical operations.

\section{Aim and objectives}
\label{sec:aim}
\subsection{Aim}\label{sec:Aim}
The aim of this thesis is to develop a behavior-based ransomware detection system in the cloud environment that is more effective than common rule-based approach.
\subsection{Objectives}\label{sec:Objectives}

To achieve the overall aim, the following objectives have been defined:

\begin{enumerate}
\item To conduct a comprehensive review of existing literature on ransomware attacks in the local and cloud environment using machine learning.

\item To create an isolated sandbox for imitation ransomware attacks in Google Workspace environment.

\item To develop a behavior-based ransomware detection model that leverages machine learning techniques.
 
\item To evaluate the performance of the proposed model in terms of its accuracy and scalability.

\item To perform a thorough analysis of the results and compare the proposed model with existing approaches for ransomware detection.

\item To provide recommendations for future work in this area and suggest potential applications for the proposed model.

\end{enumerate}
Each of these objectives is clearly defined and measurable, and will contribute to the achievement of the overall aim of the thesis. The results of the research will provide valuable insights into the behavior-based detection of ransomware attacks in the cloud environment and the potential of machine learning techniques in addressing this challenge.

\section{Research questions}
\label{sec:rq}
    \begin{enumerate}

        \item Which ML techniques have been used for ransomware detection attacks?
        \item How can the false positive rate of the machine learning-based ransomware detection system be reduced while maintaining high detection accuracy?
        \item How does the size and complexity of cloud environments impact the efficiency of machine learning-based ransomware detection?
    \end{enumerate}

\section{Method}
\label{sec:method}


In order to address the research questions of this thesis and develop a behavior-based detection system for ransomware attacks in cloud storage, we propose the following methodology, broken down into the following phases: literature review, data collection and preprocessing, model development, evaluation, and retraining.



To address RQ1, a comprehensive literature review will be conducted to identify the ML techniques that have been used for ransomware detection attacks. Relevant databases such as IEEE Xplore, ACM Digital Library, and Google Scholar will be searched for articles, conference proceedings, and other academic works. Keywords such as "ransomware detection," "machine learning," "cloud storage," "Google Drive," and "anomaly detection" will be used to search for relevant literature. The review will focus on identifying the state-of-the-art ML techniques, feature extraction methods, and the metrics used for evaluating the performance of ransomware detection models.

To create a dataset for model training and testing, we will develop an environment to emulate a ransomware attack in Google Workspace, as well as a program for running controlled ransomware attacks with different settings. We will collect log events related to Google Drive activities during the simulated ransomware attacks and during normal usage. The dataset will be preprocessed to extract relevant features and labeled to indicate whether each event is part of a ransomware attack or not.

An anomaly detection model will be developed using the dataset collected and preprocessed in the previous phase. Supervised learning ML technique and algorithms will be explored and compared (e.g., SVM, random forests). The model will be trained and fine-tuned to detect ransomware attacks with high accuracy while minimizing the false positive rate, addressing RQ2.

The performance of the developed model will be evaluated using standard metrics such as precision, recall, F1-score, and area under the ROC curve. The model's efficiency and scalability will be assessed in terms of computational complexity and resource consumption, addressing RQ3. Cross-validation techniques and other model selection strategies will be used to prevent overfitting and ensure the generalizability of the results.

To maintain the quality of the model and keep it up-to-date, a retraining strategy will be implemented. The model will be retrained weekly using a cron job, and the updated model will be deployed to Google Cloud Vertex AI. The retraining process will leverage Google Cloud Functions and Vertex AI, with triggers set up to monitor new log events related to Google Drive.


\section{Expected outcomes}
\label{sec:outcome}

By following the proposed methodology and research design, we expect to achieve the following outcomes:

Comprehensive understanding of ML techniques for ransomware detection: The literature review will provide a thorough understanding of the state-of-the-art ML techniques, feature extraction methods, and performance evaluation metrics in the domain of ransomware detection. This knowledge will be crucial in designing and implementing an efficient ransomware detection system for cloud storage.

Efficient ransomware detection model in the cloud: The development of an anomaly detection model based on the collected and preprocessed dataset is expected to yield a high-performance ransomware detection system, capable of identifying ransomware attacks in cloud storage environments with high accuracy and low false positive rates.

Insights into the impact of cloud environment size and complexity on detection efficiency: By evaluating the developed model in terms of computational complexity and resource consumption, we expect to gain insights into how the size and complexity of cloud environments may impact the efficiency of machine learning-based ransomware detection systems. This understanding will be valuable for future research and development efforts in the field of cloud security.

Robust retraining and deployment strategy: The implementation of a retraining strategy and deployment on Google Cloud Vertex AI will ensure that the ransomware detection model remains up-to-date and accurate in detecting new and evolving ransomware threats. This outcome will demonstrate the feasibility of maintaining a high-quality ML-based ransomware detection system in a dynamic threat landscape.

Contribution to the field of cloud security: By developing a behavior-based detection system for ransomware attacks in cloud storage and addressing the research questions, this project will contribute to the growing body of knowledge in the field of cloud security. The findings and insights gained from this research will be valuable for both academia and industry, helping to improve the security of cloud storage services and protect users from ransomware attacks.

\section{Time and activity plan}
\label{sec:plan}


\subsection{Data Collection (2 weeks):}
\begin{enumerate}
\item {Collect and organize the necessary data for the study}
\item {Ensure the quality and reliability of the collected data}
\end{enumerate}

\subsection {Data Analysis (3 weeks):}
\begin{enumerate}
\item {Analyze the collected data using appropriate methods}
\item {Interpret the results and draw meaningful conclusions}
\item {Prepare visual representations of the results (e.g. graphs, charts, etc.)}
\end{enumerate}

\subsection {Report Writing (3 weeks):}
\begin{enumerate}
\item {Write the main body of the report, including a literature review, methodology, results, and conclusions}
\item {Prepare the introduction and conclusion of the report}
\item {Format the report according to the required guidelines and standards}
\item {Proofread and edit the report to ensure accuracy and clarity}
\end{enumerate}

\subsection {Feedback and Revisions (1 week):}
\begin{enumerate}
\item {Obtain feedback from the supervisor on the report}
\item {Incorporate the feedback and make necessary revisions}
\item {Ensure the final report meets the requirements and standards.}
\end{enumerate}

\subsection {Final Submission (1 week):}
\begin{enumerate}
\item {Submit the final report}
\item {Prepare for the oral examination and defense of the thesis.}
\end{enumerate}

\subsection{Supervision plan}

We have agreed with our supervisor to virtually meet once every two weeks to track the progress of our master thesis. During these meetings, we will discuss the current status of our project and address any questions or issues that may arise. We will also be utilizing Github to track our progress and share the code with  supervisor. This way, he can easily keep track of my commited changes and provide feedback in a timely manner.

\section{Limitations and risk management}
\subsection{Limitations and Mitigation Strategies}

The following are the limitations of this thesis, and the corresponding mitigation strategies that have been identified:

\begin{enumerate}
\item Limitation: The proposed model is limited to detecting only known ransomware variants, and may not be able to detect new, previously unseen variants.
\begin{itemize}
\item Mitigation Strategy: Regular updates to the training data of the proposed model will be performed to ensure its ability to detect new ransomware variants.
\end{itemize}
\item Limitation: The proposed model is based on the assumption that the behavior patterns of ransomware are distinguishable from those of benign software, which may not always be the case.
\begin{itemize}
    \item Mitigation Strategy: To ensure that the proposed model is robust and not overly reliant on the assumption of distinguishable behavior patterns, the model will be evaluated on a diverse range of data and compared with other existing approaches for ransomware detection.
\end{itemize}
\end{enumerate}



\subsection{Risk Management}

The following are the identified risks and the corresponding mitigation strategies:

\begin{enumerate}
\item Risk: Confidential data may be at risk of exposure during the collection and analysis of data for model training and evaluation.
\begin{itemize}
\item Mitigation Strategy: Data collected for model training and evaluation will be anonymized to protect confidential information, and strict access controls will be implemented to ensure that only authorized personnel have access to the data.
\end{itemize}
\item Risk: The proposed model may result in a high number of false positives, leading to potential disruption of normal system operations.
\begin{itemize}
    \item Mitigation Strategy: Thorough evaluation of the model's performance on a diverse range of data will be performed, and its performance will be compared with other existing approaches for ransomware detection. The model's parameters will be adjusted as necessary to minimize false positives, and the results will be validated through field testing in a controlled environment. 
\end{itemize}
\end{enumerate}


% All references are stored in a separate bib-file: thesis-refs.bib
\setlength{\bibsep}{4pt}
\bibliography{thesis-refs}
\bibliographystyle{IEEEtranS}


% Please place the next three lines in comments if you do not need a signature enclosure.
% --------------------------------------------
% \appendix
% \newpage
% \section{Signature enclosure}
\todo[inline]{`Agreement'' with company/ organization plus signatures. Only needed for projects in collaboration with external companies or organizations.
Comment this appendix out in \texttt{project-plan.tex} if not needed.}
The company/~organization \textit{\theCompany} accepts the description of the ``work'' in
this document (``Project plan'').
\textit{\theCompany} also accepts to offer
the student/~students a reasonable amount of supervision in connection with the future degree project in relation to the parts of the degree project that are related to the ”work” at \textit{\theCompany}.
\textit{\theCompany} is aware that BTH is not part of any agreement with the company and does not guarantee that the ``work'' is carried out in a satisfactory manner.

The part of the degree project that is carried out by the student/~students ``on behalf of''
\textit{\theCompany} is only a commitment between the student/~students and the company.
BTH will evaluate the work in the form of documents that are submitted for evaluation.
All submitted works can be made public on request. 
According to regulations, the passed final report will be published in the document management system ``DIVA'' and as a part of this publication will then become a generally accessible public document.

This signature enclosure is not a legal agreement and is therefore not binding for any part
(BTH nor \textit{\theCompany}) but should be considered as a ``declaration of intention''
from \textit{\theCompany}.

\vspace{12mm}
Karlskrona, \today
\vspace{12mm}

\rule{10cm}{1pt}

Signature by supervisor at \theCompany
\vspace{12mm}

\rule{10cm}{1pt}

Printed name

% --------------------------------------------

\end{document}
