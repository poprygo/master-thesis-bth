%
% Project plan for degree project
% Template version: Don't forget to update when updating the template
% Update also the Word template. Keep the version numbers in both formats in sync.
\newcommand{\theVersion}{1.6 -- June 12, 2020}
%
\documentclass[12pt,a4paper,twoside]{article}
\usepackage{times}
%\usepackage{mathptmx}
\usepackage{multirow}
\usepackage{hyperref}
\usepackage[utf8]{inputenc}
\usepackage[T1]{fontenc}
\usepackage[top=2.5cm,bottom=2.5cm,left=2.5cm,right=2.5cm]{geometry}
% --------------------------------------------
% package "todonotes" is used for notes and comments
% To disable all notes you can use the option "disable", see second row below:
\usepackage[color=blue!10,textsize=footnotesize,textwidth=25mm]{todonotes}
%\usepackage[disable]{todonotes} %passive=do not show
% --------------------------------------------
\usepackage[sort&compress]{natbib}
\setcitestyle{numbers,square,comma}
\usepackage{enumitem}
\setlist{topsep=1ex,itemsep=0.5ex,parsep=0pt,partopsep=0pt}  % this sets the vertical spacing between list items and surrounding parapgraphs

% Redefine the course code and name.
\newcommand{\theCourse}{DV2572: Masters thesis course in Computer Science}

% Choose one of the following for the purpose of the submission.
%\newcommand{\thePurpose}{Project idea} % Use for a project idea
\newcommand{\thePurpose}{Project plan} % Use for a project plan

% Redefine the company name as a variable for easier reference.
\newcommand{\theCompany}{CompanyOrganizationName}

% *** Do not touch the following lines. BEGIN. ***
\title{\thePurpose\ for degree project\\\vspace{1mm}\small{Template version \theVersion}}
\author{\textsc{\theCourse}}
\date{\today}
\begin{document}
\maketitle
\vspace*{-5mm}
% *** END. Do not touch the lines above. ***

% *** PLEASE START HERE ***
\noindent % Do not delete this line or add an empty line below.
\begin{tabular}{|l|l|p{9.3cm}|}
\hline
Title      		  & \multicolumn{2}{|p{12cm}|}{Behaviour-based malware detection using reinforcement learning} \\
\hline
Classfificatoin      		  & \multicolumn{2}{|l|}{Provide up to 3 ACM categories*} \\
\hline\hline
\multirow{4}{*}{Student 1}
 & Name           & Yaroslav Popryho\\\cline{2-3}
 & E-Mail         & yapo22@student.bth.se \\\cline{2-3}
 & Person nr      & 991102T476 \\\cline{2-3}
 & Program        & Computer Science \\\cline{2-3}
\hline
\multirow{4}{*}{Student 2}
 & Name           & Leonid Popryho\\\cline{2-3}
 & E-Mail         & lepo22@student.bth.se \\\cline{2-3}
 & Person nr      & 991102T450 \\\cline{2-3}
 & Program        & Computer Science \\\cline{2-3}
\hline
\multirow{3}{*}{Supervisor}
 & Name \& title  & Oleksandr Adamov, University lecturer \\\cline{2-3}
 & E-Mail         &  oleksandr.adamov@bth.se \\\cline{2-3}
 & Department     &  Department of Software Engineering \\
\hline

\end{tabular}

\vspace{1mm}
\noindent
*2012 ACM Computing Classification System: www.acm.org/about/class/2012

\noindent
**Co-advisor from industry or a higher education institution (HEI).

\todo[inline]{Don't forget to delete the blue comment-boxes before submission.
See \textbackslash usepackage-rows in the LaTeX-code.}
\todo[inline]{You need to include and cite some references that are relevant for your work,
at least in Section \ref{sec:intro} and Section \ref{sec:method}. This can include references to important and/or recent works in the area regarding key concepts and terminology, related work, research designs, etc. You do not need to cite the course literature \cite{berndtsson2007thesis,evans2014write,blomkvist2014metod,host2006att,trochim2015research}. These are added here just to show you how the citation marks and references should look like. Note that references are numbered in alphabetic order. More information about academic writing can be found in Evans et al.'s book \cite{evans2014write}.}


\section{Introduction}
\label{sec:intro}

Behavior-based malware detection is a crucial area in the field of computer security, as the number of cyber attacks continues to increase and malware becomes more sophisticated. To address these challenges, the use of reinforcement learning for behavior-based malware detection has emerged as a promising solution. Reinforcement learning is a type of machine learning that focuses on learning from interactions with the environment and making decisions based on those experiences.

This thesis will focus on exploring the use of reinforcement learning for behavior-based malware detection. A review of existing research will be conducted to understand the current state of the field and identify any limitations in the existing methods for behavior-based malware detection. Based on the review, a gap in the current methods for behavior-based malware detection will be identified and the specific problem to be addressed in this thesis will be to investigate the potential of reinforcement learning to address these limitations.

The problem is relevant to the field of computer security, as it has the potential to improve the effectiveness of malware detection and enhance the overall security of computer systems.
\textit{
In the introduction, you should describe the field you are working in and eventually lead the reader to the specific problem that you intend to address in your thesis. It is important that your presentation is based on established knowledge and that it clearly identifies a relevant problem that is of general interest.}

\textit{Typically, an introduction proceeds from the general to the specific. As a rough guideline, you can think of the following things that an introduction should do (not necessarily in the exact order described below).
\begin{enumerate}
    \item Establish the importance of the (general) field.
    \item Provide relevant background facts or information.
    \item Define key terminology (if required).
    \item Present the problem area or research focus, i.e. moving from the general field (see 1.) to the specific area you are dealing with.
    \item Give a brief overview over existing research/~contributions in the specific area you are dealing with.
    \item Identify a gap in this work.
    \item Describe the specific problem you will address (as a result from 6.).
    \item Explain in which ways the problem (as described in 7.) is relevant.
\end{enumerate}
}

\subsection{Ethical, societal and sustainability aspects}
\label{sec:ethics}

\textit{In the project plan, you should briefly discuss aspects regarding ethics, society and sustainability. If you think there are no such aspects that are relevant for your work, please argue why this is the case. Please note that subsections should be formatted and numbered like this one. The paragraphs of sections, subsections and subsections are not indented.}

\subsubsection{Ethics}\label{sec:Ethics}
    \begin{enumerate}

        \item Privacy concerns: The use of behavior-based malware detection techniques may involve collecting and analyzing data from the user's device, which could potentially compromise their privacy. It's important to consider the ethical implications of collecting and storing this data and to ensure that it's handled securely and in accordance with privacy laws.

        \item  Bias: Reinforcement learning algorithms can be biased if the training data used to develop them contains bias. It's important to consider the potential for bias in the training data and to ensure that the algorithms are fair and unbiased in their predictions.

    \end{enumerate}

\subsubsection{Society}\label{sec:Society}
    \begin{enumerate}

        \item Accessibility: The deployment of behavior-based malware detection techniques should be accessible to a broad range of users, including those with disabilities. It's important to consider the accessibility of the technology and to ensure that it can be used by as many people as possible.

        \item  Societal impact: The use of behavior-based malware detection techniques may have a societal impact, for example, by contributing to the spread of fear about cybersecurity or increasing the surveillance of individuals. It's important to consider the potential impact on society and to ensure that the technology is used in a responsible and ethical manner.

    \end{enumerate}

\subsubsection{Sustainability}\label{sec:Sustainability}
    \begin{enumerate}

        \item Energy Efficiency: The behavior-based malware detection system may consume a significant amount of energy, which can have a negative impact on the environment. It is important to consider the energy consumption of the system and ensure that it is designed and deployed in an energy-efficient manner.

        \item  Resource Utilization: The deployment of the system may require a substantial amount of computing resources. It is essential to consider the utilization of resources and make sure that they are used in an efficient and sustainable manner.

        \item Environmentally Responsible Disposal: The end-of-life disposal of the technology used for behavior-based malware detection should be handled in an environmentally responsible way to prevent any potential harm to the environment.
    \end{enumerate}


\section{Aim and objectives}
\label{sec:aim}

The aim of this project is to evaluate the efficacy and limitations of using reinforcement learning for behavior-based malware detection, and to explore potential ways to improve its performance and robustness.

Objectives:
\begin{enumerate}

    \item To conduct a comprehensive literature review of existing studies on behavior-based malware detection and reinforcement learning.
    \item To implement and evaluate a reinforcement learning-based behavior-based malware detection system using relevant datasets.
    \item To compare the performance of the reinforcement learning-based approach with traditional behavior-based malware detection methods.
    \item To identify and analyze the impact of various design choices, such as reward function, feature selection, and representation, on the performance of the reinforcement learning-based malware detection system.
    \item To investigate potential methods for improving the generalizability and robustness of the reinforcement learning-based approach in the face of new and evolving threats.
    \item To examine the feasibility and benefits of using reinforcement learning for dynamic adaptation of the behavior-based malware detection system to changes in the operating environment.
    \item To address the interpretability and explainability challenges of reinforcement learning-based malware detection for stakeholders in the cybersecurity domain.
\end{enumerate}

\textit{
This section should contain concise and clear descriptions of what you want to achieve with this thesis (aim), and how this overall aim is broken down into measurable and operationable objectives. 
}



\section{Research questions}
\label{sec:rq}
    \begin{enumerate}
    
        \item How effective is reinforcement learning in detecting malware based on its behavior
        compared to traditional signature-based methods?
        \item Can reinforcement learning algorithms improve the accuracy and speed of behavior-
        based malware detection in real-time environments?
        \item What are the limitations of using reinforcement learning for behavior-based malware
        detection and how can they be addressed?
        \item (Optional) Can reinforcement learning be used to dynamically update malware detection models
        based on changing malware behavior patterns?
        \item (Optional) How does the choice of reinforcement learning algorithm impact the effectiveness of
        behavior-based malware detection?
        \item (Optional) How does the choice of reward function affect the performance of reinforcement learning-based malware detection?
        \item (Optional) What is the impact of feature selection and representation on the accuracy of reinforcement learning-based malware detection?
        \item (Optional) How can the generalizability and robustness of reinforcement learning-based
        malware detection be improved in the face of new and evolving threats?
    \end{enumerate}

\section{Method}
\label{sec:method}
This section should describe the research design or methods you intend to use to gather the evidence (data) you need to solve the problem or answer the research questions, and explain why these
methods are appropriate. If you intend to measure certain properties or attributes, you
need to explain why these properties or attributes are relevant and how they could be measured
in a reliable way.

Please make sure to discuss whether there are aspects in your work that might require
approval by an ethics committee (see Section \ref{sec:ethics}). This is particularly important
when your research involves experiments with humans.


\section{Expected outcomes}
\label{sec:outcome}
This section should briefly describe what you expect as an outcome and in which ways this outcome is important/helpful and for whom. 


\section{Time and activity plan}
\label{sec:plan}
This section should provide information about project planning and project tracking.
Project planning means a list of tasks that will be carried out to reach your aim,
and how these tasks are placed in time, i.e. a \textit{work breakdown structure}
and a \textit{schedule}.
Project tracking means a description of how you intend to follow up your work and make it possible to check whether you are ``on track''.

Please include activities beyond just preparing and submitting the required documents.
A trivial work breakdown structure like ``planning, data collection, analysis,
report writing'' is insufficient. You don't need to prepare a detailed weekly
schedule, though, but you must try to be realistic.
Make sure to include some slack for taking care of feedback and unexpected delays.

\subsection{Supervision plan}
Project tracking and follow-up should, among others, include a brief description of how you and your supervisor plan to organize the supervision.

Briefly describe the ``way-of-working'' you have agreed upon with your supervisor and how often you intend to meet (physically or virtually). For example, state how and when you intend to inform your supervisor about your progress and issues. 

When you discuss your work and document drafts with your supervisor, consider that feedback cannot be requested outside regular office hours, \emph{even though submission deadlines might be scheduled on a Sunday}. Supervisors should give feedback in a reasonable time-frame. Planning and adhering to internal draft deadlines help you to receive quality feedback, on time. 


\section{Limitations and risk management}
\label{sec:risks}
This section should provide a description of the limitations of your
work, and a brief analysis of \textit{risks}, i.e. what might go wrong. For each risk that
is either serious or likely to happen, you should have a \textit{mitigation plan}, i.e.
what you do to prevent risks and what do you do when they actually happen.
Make sure to think about non-trivial risks.


% All references are stored in a separate bib-file: thesis-refs.bib
\setlength{\bibsep}{4pt}
\bibliography{thesis-refs}
\bibliographystyle{IEEEtranS}

% Please place the next three lines in comments if you do not need a signature enclosure.
% --------------------------------------------
\appendix
\newpage
\section{Signature enclosure}
\todo[inline]{`Agreement'' with company/ organization plus signatures. Only needed for projects in collaboration with external companies or organizations.
Comment this appendix out in \texttt{project-plan.tex} if not needed.}
The company/~organization \textit{\theCompany} accepts the description of the ``work'' in
this document (``Project plan'').
\textit{\theCompany} also accepts to offer
the student/~students a reasonable amount of supervision in connection with the future degree project in relation to the parts of the degree project that are related to the ”work” at \textit{\theCompany}.
\textit{\theCompany} is aware that BTH is not part of any agreement with the company and does not guarantee that the ``work'' is carried out in a satisfactory manner.

The part of the degree project that is carried out by the student/~students ``on behalf of''
\textit{\theCompany} is only a commitment between the student/~students and the company.
BTH will evaluate the work in the form of documents that are submitted for evaluation.
All submitted works can be made public on request. 
According to regulations, the passed final report will be published in the document management system ``DIVA'' and as a part of this publication will then become a generally accessible public document.

This signature enclosure is not a legal agreement and is therefore not binding for any part
(BTH nor \textit{\theCompany}) but should be considered as a ``declaration of intention''
from \textit{\theCompany}.

\vspace{12mm}
Karlskrona, \today
\vspace{12mm}

\rule{10cm}{1pt}

Signature by supervisor at \theCompany
\vspace{12mm}

\rule{10cm}{1pt}

Printed name

% --------------------------------------------

\end{document}
